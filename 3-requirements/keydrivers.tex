% !TEX root = ../report.tex
\section{Key-drivers}
The key drivers of this system are:
\begin{itemize}
	\item \req{kd}. Usability
	\item \req{kd}. Reliability
	\item \req{kd}. Compatibility
\end{itemize}


\begin{description}

\item [Usability] has to be the main focus of the system. Users will want to stick to using our system if the usability is better then similar systems of the competitors. Even if competitors have a better availability and or even reliability, there is a good chance that customers will stick to this system if the usability is better. 

\item [Reliability] is very important, because for this kind of system the trust everyone has is crucial. Making the stakeholders trust the system is the most important aspect of this system. If the system at some points does not provide reliable data, without specifically informing about it, then the all the information from the system will be useless.
This not necessarily mean that the system has to be very precise in calculating the statistics. It just has to be very clear about how inaccurate the data is.

\item [Compatibility] is important because in order for the system to be useful to the user, it has to collect and process the energy consumption of devices. All the different homes have different devices who each have a different set of relevant statistics to be calculated.
The sensor data will be monitored with a energy monitoring plug. However, to be able to receive valuable statistics from this data, the system needs to be able to cope with different types of devices.
%
%if the system isn't up at the times when it should, users will lose trust in the system. If a user set an alarm on a device, alarming them if the device exceeds a certain energy consumption, then ideally that alarm has to go any time the device indeed exceeds the given limit of energy consumption.

\end{description}