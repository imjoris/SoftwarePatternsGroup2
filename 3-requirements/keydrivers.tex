% !TEX root = ../report.tex
\section{Key-drivers}
The key drivers of this system are:
\begin{enumerate}
\item Usability
\item Reliability
\item Compatibility
\end{enumerate}

\begin{description}

\item [Usability] has to be the main focus of the system. Users will want to stick to using our system if the usability is better then similar systems of the competitors. Even if competitors have a better availability and or even reliability, there is a good chance that customers will stick to this system if the usability is better. 

\item [Reliability] is very important, because for this kind of system the trust everyone has is crucial. Making the stakeholders trust the system is the most important aspect of this system. If the system at some points does not provide reliable data, without specifically informing about it, the entire system is useless.\\
This not necessarily mean that the system has to be very precise in calculating the statistics. Even if system might be off by 30\%, as long as the user is aware of this and it never exceeds this marge of 30\%, it is fine. The increases and decreases of the energy consumption will still be clear.

This includes Availability, which is crucial because if the system isn't up at the times when it should, users will lose trust in the system. If a user set an alarm on a device, alarming them if the device exceeds a certain energy consumption, then ideally that alarm has to go any time the device indeed exceeds the given limit of energy consumption. Because of maintenance to the system or dependent systems, the up time of the system might not be 24/7 at all times. But if the system is not up at a certain point, the users have to know about this scheduled downtime.

\item [Compatibility], which includes Interoperability is also a very important quality attribute for the system, since the system needs to be able to operate with a range of different clients (i.e. different devices/sensors sending the electricity usage data). 

\end{description}