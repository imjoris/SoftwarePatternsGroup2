%!TEX root = ../report.tex
\newpage
\section{Risk assessment}
\label{sec:risk-assesment}

The system is confronted by several risks which are determined and mitigated in this section.
Taking those risks into account allows to avoid them or at least reduce their impact. 
The risk management involves the identification of the risks, their probability and potential impact or consequences.

The tables below explain the meaning of the definition for probability and consequence.
\begin{figure}[H]
	\centering
	\begin{tabular}{|c|c|}
		\hline \textbf{Probability} & \textbf{Likelihood of occurrence} \\ 
		\hline High                 & 0.65 - 1.00                       \\ 
		\hline Medium               & 0.35 - 0.65                       \\ 
		\hline Low                  & 0.00 - 0.35                       \\ 
		\hline
	\end{tabular} 
	\caption{The different probabilities used to classify risks}
	\label{table:risk-probability}
\end{figure}

\begin{figure}[H]
	\centering
	\begin{tabular}{|l|p{15.5cm}|}
		\hline \textbf{Severity} & \textbf{Explanation}                                                                                                                        \\ 
		\hline Severe            & A risk that can lead to loss of live or casualties.                                                                                         \\ 
		\hline Significant       & A risk that can lead to damages, can delay the project more than 3 months or causes one of the high-level requirements not to be fulfilled. \\ 
		\hline Moderate          & A risk that can lead to one of the high-level requirements not to be fulfilled to an acceptable level.                                      \\ 
		\hline Minor             & A risk that can lead to one of the high-level requirements not being fully fulfilled, but still fulfilled in an acceptable level.           \\
		\hline
	\end{tabular} 
	\caption{The different severities used to classify risks}
	\label{table:risk-severity}
\end{figure}


\subsection{Technical}
%\risk{T}
%{The energy measurements provided by the data center are wrong.} 
%{Low}
%{}
%{}
%{}
%{risk:wrong-measurement}
 % AF: who provides the data ? 
 %  WM: i think the devices in the house provide the data

\risk{T}
{The statistics provided by the data processing framework are wrong}
{Low}
{Moderate. If the statistics computed by the system are not accurate, this will lead to loss of thrust of the end user in the system.}
{Make sure the algorithms used to compute the statistics are correct.}
{Correct the algorithm, if the change is significant also inform end users about the error.}
{risk:wrong-statistics}

\risk{T}
{The data center storing the energy measurements becomes unavailable}
{Low}
{Significant}
{Store the data in a redundant way, preferably in multiple data centers, so that one data center going offline does not lead to downtime of the system.}
{If the data storage does become unavailable, new incoming data should be cached so it is not lost and users should be informed in the web interface that viewing the statistics is temporarily unavailable.}
{risk:unavailability}

\risk{T}
{Somebody gains unauthorized access to someone else's data}
{Low}
{Significant. Data about electricity usage can be used, e.g. to derive when people are home. Unauthorized data access will lead to a loss of thrust in the system by consumers.}
{Make sure access to the data requires authentication using a password at all time. Enforce users to use a strong password.}
{Make sure the unauthorized access is removed. Inform end users about which data was accessed by the unauthorized party.}
{risk:unauthorized-access}

\subsection{Business}
\risk{B}
{Wrong estimation of the budget}
{Medium}
{Significant. The final product does not have the features expected.}
{The team needs an accountant or at least someone taking care of the follow-up of the money. Make sure there are regular evaluations to keep track of the money flow.}
{Remove some requirements or features of the product, or change the hardware components used.}
{risk:budget}

\risk{B}
{Wrong estimation of the budget: the money invested in the fabrication and achievement of the product/system is not covered by the sales (shortfall/deficit)}
{Medium}
{Moderate. Stopping the sale}
{The team needs an accountant or at least someone taking care of the follow-up of the money.}
{Adding more features to the product in order to make it more competitive in the market.}
{risk:shortfall}

\subsection{Schedule}


\risk{S}
{The project is not finished at the deadline}
{Low}
{Significant. Pressure for all the team members, loss of credibility regarding the customers, selling a product with less features than expected.}
{ SRA , Schedule Risk analysis : Estimation of the duration of the project by its manager ( with the use of probability and statistics ) . Meeting for the team members every week to keep track of the timing and take decisions according to the deadline. }
{ Postpone the deadline or remove some features when the deadline can't be postponed. }
{risk:schedule}