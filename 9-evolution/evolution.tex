%!TEX root = ../report.tex
\chapter{System evolution}
\label{ch:evolution}
Energy Monitoring is a field which always grows because of the new environmental issues, such as global warming. This adds opportunities for the HEMS to grow by adding new features. This chapter describes the system evolution that is targeted to be achieved within three to five years.

\section{Monitoring other energy consumption}
As can be seen on Figure \ref{fig:home-energy-consumption}, heating is one other major energy consumption of a particular house. In the first version of the HEMS, only electricity consumption is measured. In the next release, the HEMS will also support gas consumption, because it is main energy source for space and water heating. The same features will be applied in this measurement, such as predicting upcoming gas bill and analytics regarding the usage.

\section{Native mobile application support}
Current interactions are only delivered by web interface. Although this interface supports computer (large screen) and mobile device (small screen), native mobile application has richer user interactions. Native mobile application, both in Android and iOS operating system, will be developed in order to expand this system capabilities. 

\section{Integration with payment company}
Users are only able to see electricity consumption and predict upcoming electricity bill in this first version of the HEMS. Future version will also be able to pay the electricity bill directly through the system. The HEMS will be expanded to integrate with third party payment company to make this possible.

\section{Building the HEMS own sensor}
Currently, sensors installed in the houses are not the concern of the HEMS, because the first version is aimed to build robust platform to support energy measurement in cloud environment. Although the system also has support to any third party sensors by exposing REST API, specialized sensor build for the HEMS will deliver better support, because latest updates about the system will be firstly compatible and tested with the own sensor of the HEMS. The future release of the HEMS will also include its own sensor.