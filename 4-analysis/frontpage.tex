%p339 in POSA
%= 371

%\begin{description}
%\item [Source]~\\
%Pattern-oriented Software Architecture - Volume 4 \cite{wiley-4} P.339
%\EAA  \cite{Fowler:2002:PEA:579257} P.344
%
%\item [Issue]~\\
%\item [Assumptions/Constraints]~\\
%\item [Positions]~
%%\begin{enumerate}
%%\item Publisher-Subscriber
%%\item Broker
%%\item Message Queuing
%%\item Remote Procedure Call
%%\end{enumerate}
%~\\[-1.5cm]
%\item [Decision] ~\\
%\item [Argument]~\\
%\item [Implications]~\\
%\item [Related requirements/decisions]~\\
%\end{description}


% %\myworries{
% %On p 99:
% %\begin{itemize}
% %\item Given a free choice, I'd recommend Page Controller (p333). 
% %\item More complex navigation and UI lead you toward a Front Controller (344)
% %\end{itemize}
% %}
% Page controller: controller for each page. So upon receiving a request do:
% \begin{itemize}
% \item decode url, extract data
% \item invoke model to process data
% \item determine view and use the model data to create the HTML to return
% \end{itemize}

% Front page controller:\\
% One controller for all requests/views. This allows building a filter chain, handling authentication, logging etc.
% 

% The front page will be used, because it provides more functionality to the system. The only advantage of a page controller compared to the front controller is that it has a more natural structure.



\begin{description}
\item [Source]~
\begin{itemize}
\item Pattern-oriented Software Architecture - Volume 4 \cite{wiley-4} P.339
\item \EAA  \cite{Fowler:2002:PEA:579257} P.344
\end{itemize}

\item [Issue]~\\
All the different kinds of HTTPS requests that can be made to the system have similar initial handling. If every request has a dedicated handler, the code for handling these similar functions would be duplicated. This decreases the reliability  of the system and might have negative affects on the availability as well.

\item [Assumptions/Constraints]~
\begin{itemize}
\item The users use a web browser to view a graphical interface
\item There is a central system handling all the HTTPS requests
\end{itemize}

\item [Positions]~
\begin{enumerate}
\item Page controller
\item Front controller 
\end{enumerate}

~\\[-1.5cm]
\item [Decision] ~\\
The front page provides more functionality to this system. The front page resides in the service layer. Here it will create a pipe of filters and functions using an Intercepting Filter (\cite{eaa}). This will allow the pipe to be created using a decorator pattern (\cite{eaa}) The front page will be configured to handle the logging, authentication and initial security of the request. Finally the front page dispatches the requests to the domain using the command pattern (\cite{eaa}). 

%\ign{(P346)}
\item [Argument]~
\begin{enumerate}
\item By using the front controller, a pipeline can be created that handles the similar activities in handling requests, having all the similar request functionality in one central place.
\item It simplifies configuring the system.
\item Front page reduces concurrency issues, because a new command object is created for each request. Reducing thread-safety concerns. The model, however, can have shared objects that do require thread safety management.
\end{enumerate}

\item [Implications]~\\
The page controller is more intuitive. Having a controller for each request is a clear way of structuring a web server. It increases the understanding of each single request separately, however it decreases the overall understanding because each request could be handled completely different.

\item [Related requirements/decisions]~\\
\ref{kd:2}, \ref{kd:3}, \ref{fr:receive-usage}

\end{description}

%\section{Command pattern}
%\begin{description}
%\item [Source]~\\
%Pattern-oriented Software Architecture - Volume 4 \cite{wiley-4} P.339
%\EAA  \cite{Fowler:2002:PEA:579257} P.344
%
%\item [Issue]~\\
%\item [Assumptions/Constraints]~\\
%\item [Positions]~
%%\begin{enumerate}
%%\item Publisher-Subscriber
%%\item Broker
%%\item Message Queuing
%%\item Remote Procedure Call
%%\end{enumerate}
%~\\[-1.5cm]
%\item [Decision] ~\\
%\item [Argument]~\\
%\item [Implications]~\\
%\item [Related requirements/decisions]~\\
%\end{description}
