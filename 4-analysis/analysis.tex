% !TEX root = ../report.tex
\chapter{Analysis}
\label{ch:analysis}
The following section describes the architectural decisions related to the patterns applied in the system.
In order to document our decisons we followed the publication: \textit{Using Patterns to Capture Architectural Decisions}.
%All layers (with their patterns) is very clearly described in ch8 starting on p95.

\newcommand{\EAA}{Patterns of Enterprise Application Architecture}

\section{Layers}
\manuallabel{sec:layers}{Layers}

\begin{description}
\item [Source]~\\
Pattern-oriented Software Architecture - Volume 1, P.31 \cite{wiley-1}

\item [Issue]~\\
The system consists of high-level components (e.g. user interface), which are dependent on lower level components (domain logic, database). Decoupling these components is important.

\item [Assumptions/Constraints]~
\begin{itemize}
\item It is assumed that the system will have higher-level components that depend on lower-level components.
\end{itemize}

\item [Positions]~
\begin{enumerate}
\item Relaxed Layered System
\item Layers
% Indirection Layers??
\end{enumerate}

\item [Decision] ~\\
The system will use the Layers pattern to divide the application in multiple layers. The top layer will implement the Presentation logic. Below that will be the Service Layer pattern, the third layer will implement the domain logic (domain model) and the bottom layer will be responsible for the data storage.

\item [Argument]~
\begin{enumerate}
\item The relaxed layered system is a variant of the Layers pattern which allows layers to use services of any layer below it, instead of only the next layer. This increases the flexibility and performance of the system. However, this pattern has a large negative impact on the maintainability.

\item 
Using the layers pattern increases maintainability by decoupling components of different levels of abstraction.

\end{enumerate}

\item [Implications]~\\
Using layers will have a positive impact on the maintainability and re-usability of the system. An increased maintainability will help prevent bugs, which on the longer term helps to increase the reliability of the system. 

The performance of the system will have some negative impact, since the request have to pass all the layers, even if they only need logic/data in the bottom layer.


\item [Related requirements/decisions]~\\
\ref{sec:service-layer}, \ref{sec:broker}

\end{description}

\section{Service Layer}
\manuallabel{sec:service-layer}{Service Layer}
\begin{figure}[H]
\caption{Service Layer}
\centering
\includegraphics[scale=0.5]{4-analysis/images/ServiceLayer.jpg}
\end{figure}

\begin{description}
\item [Source]~\\
Patterns of Enterprise Application Architecture, P.133 \cite{Fowler:2002:PEA:579257}

\item [Issue]~\\
The system will have different interfaces with different kinds of clients. For example, commands to compute statistics can come from the user interface, but also from the alerting module. These different interfaces have common interactions with the system to invoke the business logic.

\item [Assumptions/Constraints]~
\begin{itemize}
\item The application's business logic and/or data are accessed using several interfaces.
\end{itemize}

\item [Positions]~
\begin{enumerate}
\item Domain Model % just domain model not good enough
\item Service Layer
\end{enumerate}


\item [Decision] ~\\
The system will use the Service Layer pattern to define the boundary of the application using a layer of services.

The service layer exposes a set of services to be used by clients and for each service, there is a certain script that will be executed when the service is called. The service layer will be used with the "operation script" variation. This means that the Service Layer consists of thick classes that contain logic.
%TODO explain the figure
\item [Argument]~
\begin{enumerate}
\item Using just the Domain Model pattern alone does not make a distinction between application logic and domain logic.

\item The service layer does allow making a distinction between the application- and domain logic, which 
\end{enumerate}

\item [Implications]~\\
There is a boundary between the application- and domain logic. 

As a consequence of using the Service Layer pattern, the functionalities that the system offers have to be decomposed into services. \\
There will be a service for collecting the electricity usage data and computing the statistics. There will also be a service for configuring the system (adding new devices, creating alerts etc.).

\item [Related requirements/decisions]~\\
\ref{sec:layers}, \ref{INT:1}, \ref{INT:2}, \ref{fr:interface-selectstats}, \ref{fr:receive-usage}

\end{description}


%\section{Layer Supertype}
%
%\begin{description}
%%
%%Context – a recurring set of situations in which the
%%pattern applies.
%%Problem - a set of forces (goals and constraints)
%%occurring in this context.
%%• Forces are often competing
%%Solution - a canonical design form or rule that
%%someone can apply to resolve these forces.
%%• The solution balances the forces.
%
%%\begin{description}
%%\item [Context]
%%\item [Problem]
%%\item [Solution]
%%\item [Source]
%
%\item [Context]
%There is duplicate code within each layer of the system.
%
%\item [Problem]
%Duplicate code makes the system hard to modify and maintain.
%
%\item [Solution]
%\item [Source]

%\item [Context]~\\
%There is duplicate code within each layer of the system.

%\item [Problem]~\\
%Duplicate code makes the system hard to modify and maintain.

%\item [Solution]~\\
%With the Layer Supertype pattern, all the classes of a certain layer have the same super class. This super class then contains the features that are very common for the layer.
%
%This pattern will be used in every layer.
%\begin{description}
%\item[Service layer] All the services need to take care of security. The client needs to be authenticated and the data needs to be decryption by the service layer. All this security logic will be placed in a super type using the "Layer Supertype" pattern In the service layer it contains the security logic.
%
%\item[Domain layer] In this layer, the sypertype will contain common features and functions for handling storage.
%%\myworries{From book:
%%-  "Common features, such as the storage and handling of 'Identity Fields (216), can go there."
%
%\item[Data source layer]
%The data mapper in the data source pattern can use a layer super type that handles all the common behavior, which can greatly reduce the extra work of coding. 

%\item [Context]
%There is duplicate code within each layer of the system.
%
%\item [Problem]
%Duplicate code makes the system hard to modify and maintain.
%
%\item [Solution]
%With the Layer Supertype pattern, all the classes of a certain layer have the same super class. This super class then contains the features that are very common for the layer.
%
%This pattern will be used in every layer.
%\begin{description}
%\item[Service layer] All the services need to take care of security. The client needs to be authenticated and the data needs to be decryption by the service layer. All this security logic will be placed in a super type using the "Layer Supertype" pattern In the service layer it contains the security logic.
%
%\item[Domain layer] In this layer, the sypertype will contain common features and functions for handling storage.
%%\myworries{From book:
%%-  "Common features, such as the storage and handling of 'Identity Fields (216), can go there."
%
%\item[Data source layer]
%The data mapper in the data source pattern can use a layer super type that handles all the common behavior, which can greatly reduce the extra work of coding. 


%(p308 (Metadata mapping pattern))
%\end{description}

%\myworries{p475 POEAA
%"A type that acts as the super type for all types in the layer".}

%\end{description}

%With the Layer Supertype pattern, all the classes of a certain layer have the same super class. This super class then contains the features that are very common for the layer.
%
%This pattern will be used in every layer.
%\begin{description}
%\item[Service layer] All the services need to take care of security. The client needs to be authenticated and the data needs to be decryption by the service layer. All this security logic will be placed in a super type using the "Layer Supertype" pattern In the service layer it contains the security logic.
%
%\item[Domain layer] In this layer, the sypertype will contain common features and functions for handling storage.
%%\myworries{From book:
%%-  "Common features, such as the storage and handling of 'Identity Fields (216), can go there."}
%
%\item[Data source layer]
%The data mapper in the data source pattern can use a layer super type that handles all the common behavior, which can greatly reduce the extra work of coding. 
%
%%(p308 (Metadata mapping pattern))
%\end{description}
%
%%\myworries{p475 POEAA
%%"A type that acts as the super type for all types in the layer".}
%
%\end{description}

\section{Front controller}
%p339 in POSA
%= 371

%\begin{description}
%\item [Source]~\\
%Pattern-oriented Software Architecture - Volume 4 \cite{wiley-4} P.339
%\EAA  \cite{Fowler:2002:PEA:579257} P.344
%
%\item [Issue]~\\
%\item [Assumptions/Constraints]~\\
%\item [Positions]~
%%\begin{enumerate}
%%\item Publisher-Subscriber
%%\item Broker
%%\item Message Queuing
%%\item Remote Procedure Call
%%\end{enumerate}
%~\\[-1.5cm]
%\item [Decision] ~\\
%\item [Argument]~\\
%\item [Implications]~\\
%\item [Related requirements/decisions]~\\
%\end{description}


% %\myworries{
% %On p 99:
% %\begin{itemize}
% %\item Given a free choice, I'd recommend Page Controller (p333). 
% %\item More complex navigation and UI lead you toward a Front Controller (344)
% %\end{itemize}
% %}
% Page controller: controller for each page. So upon receiving a request do:
% \begin{itemize}
% \item decode url, extract data
% \item invoke model to process data
% \item determine view and use the model data to create the HTML to return
% \end{itemize}

% Front page controller:\\
% One controller for all requests/views. This allows building a filter chain, handling authentication, logging etc.
% 

% The front page will be used, because it provides more functionality to the system. The only advantage of a page controller compared to the front controller is that it has a more natural structure.



\begin{description}
\item [Source]~
\begin{itemize}
\item Pattern-oriented Software Architecture - Volume 4 \cite{wiley-4} P.339
\item \EAA  \cite{Fowler:2002:PEA:579257} P.344
\end{itemize}

\item [Issue]~\\
All the different kinds of HTTPS requests that can be made to the system have similar initial handling. If every request has a dedicated handler, the code for handling these similar functions would be duplicated. This decreases the reliability  of the system and might have negative affects on the availability as well.

\item [Assumptions/Constraints]~
\begin{itemize}
\item The users use a web browser to view a graphical interface
\item There is a central system handling all the HTTPS requests
\end{itemize}

\item [Positions]~
\begin{enumerate}
\item Page controller
\item Front controller 
\end{enumerate}

~\\[-1.5cm]
\item [Decision] ~\\
The front page provides more functionality to this system. The front page resides in the service layer. Here it will create a pipe of filters and functions using an Intercepting Filter (\cite{eaa}). This will allow the pipe to be created using a decorator pattern (\cite{eaa}) The front page will be configured to handle the logging, authentication and initial security of the request. Finally the front page dispatches the requests to the domain using the command pattern (\cite{eaa}). 

%\ign{(P346)}
\item [Argument]~
\begin{enumerate}
\item By using the front controller, a pipeline can be created that handles the similar activities in handling requests, having all the similar request functionality in one central place.
\item It simplifies configuring the system.
\item Front page reduces concurrency issues, because a new command object is created for each request. Reducing thread-safety concerns. The model, however, can have shared objects that do require thread safety management.
\end{enumerate}

\item [Implications]~\\
The page controller is more intuitive. Having a controller for each request is a clear way of structuring a web server. It increases the understanding of each single request separately, however it decreases the overall understanding because each request could be handled completely different.

\item [Related requirements/decisions]~\\
\ref{kd:2}, \ref{kd:3}, \ref{fr:receive-usage}

\end{description}

%\section{Command pattern}
%\begin{description}
%\item [Source]~\\
%Pattern-oriented Software Architecture - Volume 4 \cite{wiley-4} P.339
%\EAA  \cite{Fowler:2002:PEA:579257} P.344
%
%\item [Issue]~\\
%\item [Assumptions/Constraints]~\\
%\item [Positions]~
%%\begin{enumerate}
%%\item Publisher-Subscriber
%%\item Broker
%%\item Message Queuing
%%\item Remote Procedure Call
%%\end{enumerate}
%~\\[-1.5cm]
%\item [Decision] ~\\
%\item [Argument]~\\
%\item [Implications]~\\
%\item [Related requirements/decisions]~\\
%\end{description}


\section{Domain model}
%
%\begin{description}
%\item [Context]
%The domain logic consists of complex functions for serving web request and analyzing data.
%
%\item [Problem]
%The functionality of the system must be modifiable and must not contain duplicate code to prevent inconsistency.
%
%\item [Solution]
%The domain logic is complex and so it requires the use of the domain model pattern. This means that the domain is Object Oriented, with every class representing one specific, individual, meaningful part.
%This is the most advanced pattern, reducing code duplication and increasing flexibility of the system.
%\end{description}

\begin{description}
\item [Source]~\\
\EAA, P.116 \cite{Fowler:2002:PEA:579257}\\
%POSA P.182 (=219)
Pattern-oriented Software Architecture - Volume 4, P.182 \cite{wiley-4}

\item [Issue]~\\
The domain logic consists of complex functions for serving web request and analyzing data. The functionality of the system must be modifiable and must reduce the amount of duplicate code to a minimum in order to prevent inconsistencies.

\item [Assumptions/Constraints]~
\begin{itemize}
\item The presentation layer uses a front controller pattern
\end{itemize}

\item [Positions]~
\begin{enumerate}
\item Transaction Script
\item Table Module
\item Domain model
\end{enumerate}

\item [Decision] ~\\
The domain model pattern will be used.

\item [Argument]~\\
The domain logic is complex and so it requires the use of the domain model pattern. This means that the domain is Object Oriented, with every class representing one specific, individual, meaningful part. The domain model is the most advanced domain pattern, minimizing code duplication and increasing the flexibility of the system. The domain model also integrates well with the front page controller in the first layer.

\item [Implications]~\\
Having a script that is solely responsible for handling a specific request, would not be conform the domain model. This makes it harder to understand how the requests are handled, because the used functions might be spread across the classes in the domain layer.

\item [Related requirements/decisions]~\\
%\end{description}
\ref{fr:compute-total} \ref{fr:compute-timeperiod} \ref{fr:compute-bill}  \ref{fr:show-accuracy} \ref{fr:choose-alerts} \ref{kd:3}
 
%The alternatives are : Transaction Script and Table module.
%
%Transaction script: Each system call has its corresponding script that will be executed.\\
%Table module: One class per table in the database that contains all the logic acting on that data.
\end{description}

\section{Unit of work}
\manuallabel{sec:unitofwork}{Unit of Work}

\begin{description}
\begin{figure}[H]
\centering
\includegraphics[scale=0.7]{7-software/images/UnitOfWork.png}
\caption{The Unit of Work class}
\label{fig:unitofworkclass}
\end{figure}
\item [Source]~\\
\EAA , P.184 \cite{eaa}\\
Pattern-oriented Software Architecture - Volume 4, P.541 \cite{wiley-4}

\item [Issue]~\\
The system has several object stored in a database which can be edited and created. For example, new sensor data of devices becomes available and changes to the configuration (changing device names etc.) can be made. Updating database records on each change leads to a lot of database calls, which is bad for performance.

\item [Assumptions/Constraints]~\\
\begin{itemize}
\item The domain layer uses the domain model pattern.
\item Requests to store and receive data are    being made continuously while the server runs.

\end{itemize}

\item [Positions]~
\begin{enumerate}
\item Active Record
\item Unit of Work with Caller Registration
\item Unit of Work with Object Registration
\end{enumerate}

\item [Decision] ~\\
Unit of Work pattern will be used to keep track of changes to objects and to coordinate writing these changes to the database in one database call.
The variant used is the Object Registration variant, which removes the burden of having to register the object explicitly with the Unit of Work. Registering the object self could introduce bugs.

\item [Argument]~\\
\begin{enumerate}
\item With Active Record, every change to an in-memory object leads to one or multiple database calls. This reduces the performance significantly, especially when using the domain model pattern.

\item The Unit of Work instead keeps track of these changes to allow writing these changes to the database in a single call.\\ When using Unit of Work with Caller Registration, the creator of the object should register it with the Unit of Work explicitly, or else the Unit of Work will not keep track of its changes. This offers the flexibility to have changes to in-memory objects that are never written to database. 

\item Using Unit of Work with Object Registration means that the creator of the object does not need to explicitly register it with the Unit of Work. Instead, the created object is implicitly registered (e.g. as part of the logic in the constructor). 
\end{enumerate}

\item [Implications]~\\
Using Unit of Work will reduce the load on the database (the number of database calls). It does however introduce a delay before the change in the in-memory object is present in the representation in the database. \\
Callers editing the objects should call the \verb|commit| method on the object after making changes to them, so that these changes are saved to the database.

\item [Related requirements/decisions]~\\
\ref{kd:1}, \ref{fr:store-usage}, \ref{fr:retrieve-usage}, \ref{RE:2}

\end{description}

\section{Broker}
\manuallabel{sec:broker}{Broker}

\begin{description}
\item [Source]~\\
Pattern-oriented Software Architecture - Volume 4, P.237 \cite{wiley-4}

\item [Issue]~\\
The system uses several servers to compute the statistics. This introduces a lot of challenges, like communication to these servers and dividing the work over these servers. The application code should not have to address these challenges.

\item [Assumptions/Constraints]~\\
It is assumed that the workload for computing statistics can be divided over multiple nodes, and that multiple nodes are available.

\item [Positions]~
\begin{enumerate}
\item Broker 
\item Publisher-Subscriber
%\item Message Queuing %TODO is this better maybe, it allows temporary outage -> good reliability
\item Messaging
\end{enumerate}


\item [Decision]~\\
The broker pattern will be used. The broker pattern is part of the domain layer.

%from p73 (=110)
\item [Argument]~\\
\begin{itemize}
\item Using a broker creates a layer of abstraction for the specific layer using it. This increases encapsulation, location independence and scale ability. With the Broker pattern, many clients can make remote method invocations on methods hosted by the server. It is best suited for systems that want to hide the presence of the network.

\item Publish-Subscriber allows exchanging events in a one-to-many fashion. It does, however, not guarantee that an event will be processed.

\item With messaging there is also no guarantee that a message will be consumed. 
\end{itemize}


\item [Implications]~

\begin{itemize}
\item The communication between the system and the servers in the compute cluster are handled by the broker. 
\item Adding another layer of abstraction which must be used can reduce the performance of the system
\item If the broker fails, the sourced that the broker provided can't be accessed any more.
\end{itemize}

\item [Related requirements/decisions]~\\
\ref{fr:compute-total}, \ref{fr:compute-bill}, \ref{fr:compute-bill}, \ref{SEC:2}, \ref{RE:2}, \ref{sec:layers}
%Refer to layers
%Most B ROKER realizations are based on a L AYERS (185) architecture
%to manage complexity, such as CORBA [OMG04a] and Microsoft’s
%.NET Remoting [Ram02]. These layers are further decomposed into
%‘special-purpose’ components for specific networking and communi-
%cation tasks. We illustrate this partitioning using the CORBA layering
%[SC99]—other layering schemes and middleware may involve differ-
%ent assignments [VKZ04].
\end{description}

\section{Data Mapper}
\manuallabel{sec:datamapper}{Data Mapper}
%TODO add a figure
\begin{description}

\item [Source]~\\ 
%\item Pattern-oriented Software Architecture - Volume 1, P.31 \cite{wiley-1}
 Pattern-oriented Software Architecture - Volume 4, P.540 \cite{wiley-4} \\
 \EAA, P.165 \cite{eaa}
%540=577


\item [Issue]~\\
For the data source layer, there should be a way to store the data where the domain layer is shielded from the details of how the data is stored.

\item [Assumptions/Constraints]~
\begin{itemize}
\item The domain layer uses the domain model pattern
\end{itemize}

\item [Positions]~
\begin{enumerate}
\item Data mapper% (p165)
\item Table data gateway% (p144)
\item Row data gateway% (p152)
\item Active record %(p160)
\end{enumerate}

\item [Decision] ~\\
The data mapper pattern will be used. The data mapper pattern creates the most abstraction between the domain and data source layer.

\item [Argument]~\\
%Because the domain model is used in the domain layer, the data mapper pattern is best to use. \ign{(153 and 97)}
\begin{enumerate}
\item The data mapper pattern best suits the domain model pattern used in the domain layer.  This pattern is the most advanced pattern, and provides the best functionality and abstraction.
\item
\item The logic that will operate on the data will generate statistics of the data, which can become quite complex and will be stored in a different table than the device objects. 
This is why the Table- and Row Data gateway patterns are not choosen.
\item
With the active record pattern, the database access/communication, data and logic of that data is all stored in the same class. This means there is a tight coupling between the data and the domain logic. Since the logic that will be executed on the sensor could be quite complex, this pattern is not preferred.
\end{enumerate}
%The data mapper pattern best suits the domain model pattern used in the domain layer
%The logic that will operate on the data will generate statistics of the data, which can become quite complex. This is why the data mapper pattern is chosen as a pattern for connecting to the data sources. This pattern is the most advanced pattern, and provides the best functionality and abstraction.
%With the active record pattern, the database access/communication, data and logic of that data is all stored in the same class. Since the logic that will be executed on the sensor could be quite complex, this patterns will not be useful.
%A gateway of any kind leads to performance overhead, because for each call coming from the domain model pattern, a database call is made. It lacks the necessary coordination.


\item [Implications]~\\
The data mapper is used in the architecture. This means that the in-memory objects in the domain layer need not be aware of the presence of the database.
The data mapper objects will implement the details of storing the object in the database.

A gateway of any kind leads to performance overhead, because for each call coming from the domain model pattern, a database call is made.

\item [Related requirements/decisions]~\\
\ref{sec:repository},  \ref{sec:unitofwork}, \ref{fr:store-usage}, \ref{fr:retrieve-usage}

\end{description}
%
%%\begin{figure}[H]
%%\caption{Database structure draft}
%%\centering
%%\includegraphics[height=10cm]{4-analysis/images/SoftwarePatternsDatabaseDraft.jpg}
%%\end{figure}
%
%Considered patterns to connect to data sources:
%\begin{itemize}
%\item Table data gateway% (p144)
%\item Row data gateway% (p152)
%\item Active record %(p160)
%\item Data mapper% (p165)
%\end{itemize}
%
%
%With the active record pattern, the database access/communication, data and logic of that data is all stored in the same class. Since the logic that will be executed on the sensor data is very complex, this patterns will not be useful.
%
%A gateway of any kind leads to performance overhead, because for each call coming from the domain model pattern, a database call is made. It lacks the necessary coordination.
%%
%%\subsection{Table data gateway}
%%\myworries{Not needed if the repository is used (i assume)}
%%
%%The data mapper can use the table data gateway to remove the dependency on how the data is queried. Queries can be replaced with stored procedures in the Table data gateway without the data mappers having to change 
%%
%%%Using a record set (504) turning into a unit of work (184)? P98 (lost the page:()

\section{Shared Repository}
\manuallabel{sec:repository}{Repository}
%\myworries{
%P322:
%"Mediates between the domain and data mapping layers using a collection-like interface for accesing domain objects."
%\\
%P323:
%"Repository supports the objective of achieving a clean separation and one-way dependency between the domain and data mapping layers"
%\\
%P324:
%"Under the covers, Repository combines Metadata Mapping (329) with a Query object (316) to automatically generate SQL code from the criteria."
%...
%"In a large system with many domain object types and many possible queries, Repository reduces the amount of code needed to deal with all the querying that go's on."
%}

%The repository pattern mediates between the domain and data layer. The repository clients create a criteria object, specifying what kind of data is wanted. For example $criteria.equals(Device.NAME, Computer)$. Then the clients use this criteria by invoking repository.matching(criteria) to receive the data from the repository. The client just asks the data, it has no further knowledge about any interaction with any data source/data base.
%The repository pattern is used by the broker pattern to get the sources the domain layer requests. The repository clients create a criteria object, specifying what kind of data is wanted. For example $criteria.equals(Device.NAME, Computer)$. Then the clients use this criteria by invoking repository.matching(criteria) to receive the data from the repository. The client just asks the data, it has no further knowledge about any interaction with any data source/data base.


%The repository gives a lot more control over how the data is handled. The benefits are:
%\begin{itemize}
%\item Reduces code (and code complexity)
%\item Increases performance
%\item Separated domain and data layers, increasing flexibility and changeability
%\end{itemize}

%Performing analysis on the data also consists of executing complex queries on the data source. The database that executes these queries, however, might change. Or the system might decide to use multiple databases and data sources.
%Using the Repository pattern, these changes can be made fast. The repository also allows for multiple configurations to exist. So an extra repository could be created for testing purposes, only using an in-memory database to increase the test execution speed. \\


%The Shared Repository pattern will be applied and the Architectural decisions documented according to the template presented in \textit{Using Patterns to Capture Architectural Decisions} \\


% Forces ?
% Context
\begin{description}
\item [Source]~\\
\EAA, P.322 \cite{eaa}\\
Architectural Pattern Revisited - A Pattern Language, P.13 \cite{avgeriou2005architectural}

 \item [Name] ~\\ 
 Shared Repository (also called Repository)
 \item [Issue]~ \\
  The external data must be stored, updated, accessed by multiple clients and servers simultaneously and analyzed.
 \item [Assumptions/constraints]~
 \begin{itemize}
 \item It is assumed that the shared repository handles the authentication to it.
 \end{itemize}
 \item [Position]~ \\ 
 In the \textit{Shared Repository pattern} one component of the system is used as a central data store, accessed by all other independent components."\\
 When using the \textit{Active Repository Pattern} the clients are notified when there is a change is the repository which is not a feature of the system.\\
 \textit{BlackBoard Pattern} " is used in an immature domain in which no deterministic approach to a solution is known or feasible " which is not the case in the HEMS. \\
 \item [Decision]~\\ 
 The system uses the Shared Repository Pattern in order to centralize the access to the data.\\
 %\textit{Arguments:} The Shared repository patttern is suitable because \\
 \item [Implications]~ \\ 
 After the integration of the Shared Repository Pattern the system has one central repository (the database) and all clients can access it by using a loggin. \\
  This pattern ensures Reusability, Changeability, Maintainability and Integrability (cf. Chapter 7 Evaluation). \\
\item [Related decisions/requirements]~\\ 
\ref{sec:datamapper}, \ref{sec:unitofwork}, \ref{fr:receive-usage}, \ref{fr:store-usage}, \ref{fr:retrieve-usage}, \ref{SEC:2} \\

\textit{The compoments in the figure can either be the users or the computation cluster to perform the statistics}

\begin{figure}[H]
\centering
\includegraphics[scale=0.7]{4-analysis/images/SharedRepo.png}
\caption{Shared Repository}
\label{fig:unitofworkclass}
\end{figure}


 \end{description}

 %One database will be accessed by all the clients . The security in ensured by the logging through the UI.

% \section{Client user interface}

% \subsection{Controller}
% %\myworries{
% %On p 99:
% %\begin{itemize}
% %\item Given a free choice, I'd recommend Page Controller (p333). 
% %\item More complex navigation and UI lead you toward a Front Controller (344)
% %\end{itemize}
% %}
% Page controller: controller for each page. So upon receiving a request do:
% \begin{itemize}
% \item decode url, extract data
% \item invoke model to process data
% \item determine view and use the model data to create the HTML to return
% \end{itemize}

% Front page controller:\\
% One controller for all requests/views. This allows building a filter chain, handling authentication, logging etc.
% Front page is better/helps with concurrency, because a new command object is created on each request. Reducing thread-safety concerns. \ign{(P346)}The model, however, can have shared objects that do require thread safety management.

% The front page will be used, because it provides more functionality to the system. The only advantage of a page controller compared to the front controller is that it has a more natural structure.

% \subsection{View}
% %\myworries{
% %Template view (350) or Transform view (361).
% %\\
% %(P forgot):
% %Template views have the edge at the moment.
% %}

% \begin{description}
% \item [Template view] Write HTML code including markers. Replace the markers with the data when the page is requested. (Play framework)
% \item [Transform view] Convert the domain data to HTML, "transform" the domain data. Upon a request, it get the domain data, for each item in the data it looks for a appropriate "transform" to transform the data to HTML.
% \end{description}

% The template view will be used, because it is used a lot more then the transform view. Major web frameworks are based on this pattern (the play framework, laravel...). The view patterns don't have an important difference in how beneficial they are to the project.

% %Transform view avoids having too much data in the HTML, because of the transform methods creating the HTML.
% %Transform view can be tested without having a web server up.

% \subsection{Model}
% %\myworries{
% %Communication with the model: p100:
% %Preference: Having everything run in one process if you can. Else Remote Facade (p388) and DTO (p 401).
% %}
% %
% %\myworries{
% %\section{Notes only}
% %\subsection{Model-View View-Model}
% %}
% This section will describe how the Model updates the view and how the view updates the model.
% %\myworries{View update model using page controller as is already described}

% To consider:
% \begin{description}
% \item [Observer Synchronization] Synchronize multiple screens by having them all be observers to a shared area of domain data.
% \item [Separated Presentation] Ensure that any code that manipulates presentation only manipulates presentation, pushing all domain and data source logic into clearly separated areas of the program.

% \item [Presentation model] Represent the state and behavior of the presentation independently of the GUI controls used in the interface

% \item [Supervising controller] Factor the UI into a view and controller where the view handles simple mapping to the underlying model and the the controller handles input response and complex view logic.

% \item [Model view controller/presenter] 

% \end{description}


%Observer Synchronization
%%Synchronize multiple screens by having them all be observers to a shared area of domain data.
%%\url{http://www.martinfowler.com/eaaDev/uiArchs.html}
%%While Observer Synchronization is nice it does have a downside. The problem with Observer Synchronization is the core problem of the observer pattern itself - you can't tell what is happening by reading the code. I was reminded of this very forcefully when trying to figure out how some Smalltalk 80 screens worked. I could get so far by reading the code, but once the observer mechanism kicked in the only way I could see what was going on was via a debugger and trace statements. Observer behavior is hard to understand and debug because it's implicit behavior.
%
%Separated Presentation:
%%\url{http://martinfowler.com/eaaDev/SeparatedPresentation.html}
%Ensure that any code that manipulates presentation only manipulates presentation, pushing all domain and data source logic into clearly separated areas of the program.
%%Most of examples you'll see from me follow Separated Presentation, simply because I find it such a fundamental design technique
%
%Presentation model:
%%\url{http://www.martinfowler.com/eaaDev/PresentationModel.html}
%Represent the state and behavior of the presentation independently of the GUI controls used in the interface
%
%Supervising controller\\
%Factor the UI into a view and controller where the view handles simple mapping to the underlying model and the the controller handles input response and complex view logic.
%
%%MVP potel:
%%\url{Potel: http://www.wildcrest.com/Potel/Portfolio/mvp.pdf}

\section{Model-View-Controller}
Model-View-Controller (MVC) started as a framework developed by Trygve Reenskaug for the Smalltalk platform in the late 1970s \cite{Fowler:2002:PEA:579257}. Since then it has played an influential role in most UI frameworks and in the thinking about UI design.

\begin{figure}[H]
\centering
\includegraphics[scale=0.8]{4-analysis/images/analysis-mvc.pdf}
\caption{Model-View-Controller pattern illustration \cite{Fowler:2002:PEA:579257}}
\label{fig:analysis-mvc}
\end{figure}

\begin{description}
\item [Source]~\\
Pattern-oriented Software Architecture - Volume 1 P.125 \cite{wiley-1}\\
Patterns of Enterprise Application Architecture, P.330 \cite{Fowler:2002:PEA:579257}

\item [Issue]~\\
The system must handle the request from user by dynamic web interface. In order to increase re-usability, the views may be decoupled from the logic, resulting in separated logic (controller and model) and view.

\item [Assumptions/Constraints]~
\begin{itemize}
\item It is assumed that the client uses web browser to connect to the system.
\item It is assumed that there are multiple views in the system.
\end{itemize}


\item [Positions]~
\begin{enumerate}
\item Presentation-Abstraction-Control
\item Layers-Model-View-Controller
\end{enumerate}


\item [Decision] ~\\
The system will use the Model-View-Controller pattern to decouple the view and the logic behind it. This pattern resides in the service layers of this system.

\item [Argument]~
\begin{enumerate}
\item The Presentation-Abstraction-Control (PAC) pattern decompose the system into a tree-like hierarchy of agents, which each agent is made of its own presentation (UI), abstraction, and control. This type of pattern is not well suited to the HEMS since the HEMS only needs decoupled views with the same control and abstraction.

\item The Model-View-Controller (MVC) pattern only decouple the views, models, and controller, which indicates that different views may use the same controller and/or model. This pattern is well-suited with the HEMS and is good for re-usability, which we think has less complexity.

\end{enumerate}

\item [Implications]~\\
By implementing the MVC pattern, several views are created to support multiple pages. Each view may use the same controller that provides required process. This will also allows building a filter chain, handling authentication, logging etc. Thus, this will drive better reusability and have less complexity. Furthermore, it is also easier to modify the code since the view and the logic are separated.

\item [Related requirements/decisions]~\\
\ref{fr:retrieve-usage}, \ref{fr:interface-selectstats}, \ref{fr:interface-login}, \ref{fr:alerts}, \ref{fr:interface-graph} 

\end{description}



\section{Template view}
Template view pattern renders information into HTML by embedding markers in an HTML page \cite{Fowler:2002:PEA:579257}.
	
\begin{figure}[H]
\centering
\includegraphics[scale=0.8]{4-analysis/images/analysis-template-view.pdf}
\caption{Template view pattern illustration \cite{Fowler:2002:PEA:579257}}
\label{fig:analysis-template-view}
\end{figure}

\begin{description}
\item [Source]~\\
Patterns of Enterprise Application Architecture, P.350 \cite{Fowler:2002:PEA:579257}

\item [Issue]~\\
In the view component of MVC, the system must handle dynamic web interface in a Single Page Application (SPA) flavored application. The same view (HTML code) will be displayed several time in different page, which will be possibly causing code smells.

\item [Assumptions/Constraints]~
\begin{itemize}
\item This pattern is only applied on web application.
\item The system is developed using framework that supports this pattern.
\end{itemize}


\item [Positions]~
\begin{enumerate}
\item Transform view
\item Two step view
\item Template view
\end{enumerate}


\item [Decision] ~\\
The system will utilize template view in order to increase reusability and to prevent code smells in the project. In this way, several HTML code snippets that corresponds to same element of a page will be reused by embedding markers. Thus, this will lead to a better modifiability as well.

\item [Argument]~
\begin{enumerate}
\item Transform view processes domain data element by element and transforms it into HTML. In the system, such task is handled by the models. Thus, if this pattern is used, there will be duplication of work.

\item Two step view turns domain data into HTML in two steps: first by forming some kind of logical page, then rendering the logical page into HTML. These operations are considered as inefficient, as the forming logical page and rendering HTML can be joined. There is also overlapping between models, controllers, and views in this pattern.

\item Template view renders information into HTML by embedding markers in an HTML page, which solves the issues and prevents code smells.
\end{enumerate}

\item [Implications]~\\
Several HTML code snippet will be reusable, which is good for reusability and simplicity. Furthermore, code smells can be prevented using this template view.

\item [Related requirements/decisions]~\\
\ref{fr:retrieve-usage}, \ref{fr:interface-selectstats}, \ref{fr:interface-login}, \ref{fr:alerts}, \ref{fr:interface-graph}

\end{description}
