% !TEX root = ../report.tex
\section{System context}
\label{sec:system-context}

The system context diagram in Figure~\ref{fig:system-context-diagram} gives an overview of the entities that interact with the system.

\begin{figure}[H]
	\centering
	\includegraphics[keepaspectratio=true,scale=0.5]{5-system/images/SystemContext.png}
	\caption{System context diagram}
	\label{fig:system-context-diagram}
\end{figure}

\subsection{Users and roles}
\begin{description}

	\item[Home Owners] ~\\ The home owners are the main users of the system. They want to know about the electricity usage in their house. 
	
	They interact with the system by viewing statistics, configuring the system and they receive alerts if they configured the web interface to send them these. 
	
	\item[Maintainers] ~\\ The maintainers of the system will apply updates to the system and read errors that might have occurred in order to solve these.
	
\end{description}

\subsection{External systems}
\begin{description}

	\item[Electricity Usage Sensors] ~\\ The electricity usage sensors are the sensors that measure the electricity usage of the devices. These sensors work by measuring the electricity passing through a power outlet. This means that they in fact measure the electricity usage of a power outlet and not that of a device. This is relevant if multiple devices are connected to the same power outlet .
	
	\item[Web Browsers] ~\\ The web interface of the system, where the system can be configured and statistics can be viewed, is presented as a web page. Users will use a range of different web browsers (on a range of different devices) to visit this web interface. It is important that the web interface works equally well in all these different web browser.
	
\end{description}
