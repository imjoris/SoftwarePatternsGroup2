% !TEX root = ../report.tex

\clearpage
\section{Elaborated model with patterns}
This section will describe the elaborated model on the basis of the patterns used in the architecture. For each patterns, this section will describe how it is implemented and how it affects the quality attributes of the system.

\subsection{Layers pattern}
\begin{figure}[H]
\centering
\includegraphics[scale=0.45]{7-software/images/Layers.png}
\caption{The Layers}
\label{fig:layers}
\end{figure}
The system is divided into three layers. The top layer is the Service Layer, see Section~\ref{sec:service-layer-pattern} for more information about this layer.

The middle layer is the Domain Layer and is responsible for the domain logic. The Domain Model contains all the classes, has an in-memory representation of the data and contains the logic which is inherent to the objects.
It uses the Unit of Work pattern (see Section~\ref{sec:unit-of-work-pattern}) to keep track of the changes to objects, so not every change leads to a new database call. 
The other components in the Domain Layer are connected to the Domain Model, so they have access to the classes in there. 
The Alerting component is responsible for sending alerts by email to the end user, for which it depends on an external Email Gateway.\\
The Domain Model exposes an interface, which is used by the API on the Service Layer to store new sensor data. The Configuration and Statistics components also expose an interface to the Service Layer, which are used by the Web Interface, where a user can change the configuration and compute statistics.
% gateway is actually also a pattern :o

The Data Source layer contains the Unit of Work, which keeps track of the changes to the objects and translates those changes to database transactions when the object is committed. The layer also contains a Database Driver, which handles the communication with the database.

%TODO explain individual components (in one of the views?)


\subsection{Service Layer pattern}
\label{sec:service-layer-pattern}


\subsection{Unit of Work pattern}
\label{sec:unit-of-work-pattern}
\begin{figure}[H]
\centering
\includegraphics[scale=0.8]{7-software/images/UnitOfWork.png}
\caption{The Unit of Work class}
\label{fig:unitofworkclass}
\end{figure}

The Unit of Work pattern is used to keep track of the changes made to objects and newly created objects. Whenever an object is created, changed or deleted, the Unit of Work is told about this. 
Whenever the object can be saved to the database, the \verb|commit()| method of the Unit of Work is called, which translates the stored changes into database transactions.

A sequence diagram showing an example of this can be seen in Figure~\ref{fig:unitofworkseq}. Here, the StatisticsController constructs a new Device object, which is fetched from the database and then registers itself with the Unit of Work. When the StatisticsController changes the name of this device, the device object registers itself as dirty with the Unit of Work. 
When the device object is saved, it calls \verb|commit()| on the Unit of Work, which leads to the device updating the appropriate fields in the database.


\begin{figure}[H]
\centering
\includegraphics[scale=0.7]{7-software/images/UnitOfWorkSeq.png}
\caption{Sequence diagram showing an update to a Device-object using Unit of Work}
\label{fig:unitofworkseq}
\end{figure}

%It also uses an \textit{Identity Map} in order to guarantee that
  % explain this pattern somewhere

% etc...