% !TEX root = ../report.tex

\clearpage
\section{Elaborated model with patterns}
This section will describe the elaborated model on the basis of the patterns used in the architecture. For each patterns, this section will describe how it is implemented and how it affects the quality attributes of the system.

\subsection{Layers pattern}
\begin{figure}[H]
\centering
\includegraphics[scale=0.45]{7-software/images/Layers.png}
\caption{The Layers}
\label{fig:layers}
\end{figure}
The system is divided into three layers. The top layer is the Service Layer, see Section~\ref{sec:service-layer-pattern} for more information about this layer.

The middle layer is the Domain Layer and is responsible for the domain logic. The Domain Model contains all the classes, has an in-memory representation of the data and contains the logic which is inherent to the objects.
It uses the Unit of Work pattern (see Section~\ref{sec:unit-of-work-pattern}) to keep track of the changes to objects, so not every change leads to a new database call. 
The other components in the Domain Layer are connected to the Domain Model, so they have access to the classes in there. 
The Alerting component is responsible for sending alerts by email to the end user, for which it depends on an external Email Gateway.\\
The Domain Model exposes an interface, which is used by the API on the Service Layer to store new sensor data. The Configuration and Statistics components also expose an interface to the Service Layer, which are used by the Web Interface, where a user can change the configuration and compute statistics.
% gateway is actually also a pattern :o

The Data Source layer contains the Unit of Work, which keeps track of the changes to the objects and translates those changes to database transactions when the object is committed. The layer also contains a Database Driver, which handles the communication with the database.

%TODO explain individual components (in one of the views?)


\subsection{Service Layer pattern}
\label{sec:service-layer-pattern}

\subsection{Unit of Work pattern}
\label{sec:unit-of-work-pattern}

%It also uses an \textit{Identity Map} in order to guarantee that
  % explain this pattern somewhere


\clearpage
\subsection{Model-View-Controller}
\label{sec:mvc}
As mentioned in chapter \ref{ch:analysis}, MVC pattern is applied to decouple user-interface and the logic behind it. In this way, reusability is increased because the same models or controllers can be coupled with the same view. Modifiability is also increased because it becomes easier to modify a particular user interface or data model without interfering the logic, and vice-versa. Figure \ref{fig:mvc-architecture} depicts an example of MVC implementation in the HEMS.

\begin{figure}[H]
	\centering
	\includegraphics[width=0.8\textwidth]{7-software/images/mvc.pdf}
	\caption{Model-view-controller pattern implementation}
	\label{fig:mvc-architecture}
\end{figure}

Some models, views, and controllers are depicted in Figure \ref{fig:mvc-architecture}. Request handler handles incoming user request via HTTPS and routes it to the corresponding controller. Required data is then obtained through the models. Suitable views are used to provide user interface to the user. Some models, views, and controllers are presented in Figure \ref{fig:mvc-architecture}. However, there are more models, views, and controllers than those which are represented in the Figure \ref{fig:mvc-architecture}.

\clearpage
\subsection{Template View}
\label{sec:template-view}
Template view is implemented in this system to make the HTML code reusable in different pages. This will also make the view structure more simple. Code duplication can be prevented because instead of duplicating the code, the HTML will use a certain template.

\begin{figure}[H]
	\centering
	\includegraphics[width=0.7\textwidth]{7-software/images/template-view.pdf}
	\caption{Template view pattern implementation}
	\label{fig:template-view-architecture}
\end{figure}

Figure \ref{fig:template-view-architecture} shows an example of implementation of the template view pattern. Each page of the system (presented in green color) will combine several HTML code snippets (presented in blue color) together. \texttt{TopMenuBarSnippet} and \texttt{SideMenuBarSnippet} are used several times, as each page contains top menu bar and side menu bar. This is also good for expandability because the new page may just combine existing page template to create new web page.

% etc...