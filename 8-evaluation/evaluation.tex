% %!TEX root = ../report.tex

\newcommand{\bo}[1]{\textbf{#1}}

\chapter{Architecture evaluation}
\label{ch:evaluation}

In this chapter the architecture of the system will be evaluated. After taking architectural decisions and building the system the architects have to make sure that the patterns applied suit the architecture, especially the quality attributes and key drivers .%in terms of benefits and liablities.

\paragraph{Layers}
\paragraph{Service Layer}
\paragraph{Front-Controller}
\paragraph{Domain Model}
\paragraph{Model View Controller}
\paragraph{Unit of Work}
\paragraph{Shared Repository} 
%Repositories are the single point where we hand off and fetch objects. It is also the boundary where communication with the storage starts and ends.
%Mediates between the domain and the data mapping layers using a collection-like interface for accessing domain objects.

\textit{Benefits} \\

The repository gives a lot more control over how the data is handled. The benefits are:
\begin{itemize}
\item Reduces code (and code complexity)
\item Increases performance
\item Separated domain and data layers, increasing flexibility and changeability
\end{itemize}

Performing analysis on the data also consists of executing complex queries on the data source. The database that executes these queries, however, might change. Or the system might decide to use multiple databases and data sources.
Using the Repository pattern, these changes can be made fast. The repository also allows for multiple configurations to exist. So an extra repository could be created for testing purposes, only using an in-memory database to increase the test execution speed. \\

\textit{} Liabilities \\
\textbf{+ Reusability} \\
\textbf{+ Integrability} \\
\textbf{+ Modifiability Changeability} by the Separation of concerns\\