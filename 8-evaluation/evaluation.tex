% !TEX root = ../report.tex

\newcommand{\bo}[1]{\textbf{#1}}

\chapter{Architecture evaluation}
\label{ch:evaluation}

In this chapter the architecture of the system will be evaluated. After taking architectural decisions and building the system the architects have to make sure that the patterns applied suit the architecture, especially the quality attributes and key drivers.%in terms of benefits and liablities.

\section{Key-Driver Validation}
% http://www.cs.rug.nl/paris/papers/ECSA07.pdf

\subsection*{Layers}
The layers improve the maintainability of the system, they support reusability.
The performance is reduced, since the propagation of request/calls through the layers can be inefficient.\\
\textbf{Benefits \& Liabilities} ~
\begin{itemize}
\item[+] Maintainability
\item[-] Performance 
\end{itemize}

\subsection*{Service Layer}
The Service Layer pattern defines the boundary of the application using a layer of services to be used by clients. This increases the interoperability, since the set of operation offered by the service layer are available to many kinds of clients. \\
\textbf{Benefits \& Liabilities} ~
\begin{itemize}
\item[+] Interoperability 
\end{itemize}

\subsection*{Front-Controller}
By having a central point that handles all requests, all the requests get prepossessed for the domain layer in the same way. All the common functions like security and logging are handled by the front controller for each request.\\
It increases security because the security the request security is handled before any other operation in the system. And the security is handled the same for every request.\\
Reliability increases, because all the the common functions are centrally handled and so there are no duplicate implementations leading to unexpected results. \\
The front page increases maintainability because the way each request is received and initially handled can be modified in a central place in a very flexible way. It allows for easy modifications of the piping functions by using the decorator pattern.\\
A downside of the front page controller is that it is a single point of failure. If the front page controller is faulty or doesn't work, then the entire system fails. And if a function in the front page takes a long time to execute, then the performance of the entire system decreases. However, the front page controller makes testing the system easier since the functions are handled in one place.\\

In conclusion, the front page has the following benefits and liabilities:\\
\textbf{Benefits \& Liabilities} ~
\begin{itemize}
\item[+] Reliability
\item[+] Modifiability
\item[+] Security
\item[$-$] Performance
\end{itemize}

\subsection*{Domain Model}
\subsection*{Broker}
The reliability of the system increases by the use of the Broker pattern. The pattern guarantees the delivery of messages to the destination. This guarantees that statistics are computed by a server in the compute cluster.
The modifiability of the system increases, since the brokers separate the messaging logic from the domain logic. This means that changes to the messaging logic do not affect the rest of the system.

The interoperability of the system increases as well, since the broker allow the system to interoperate with the servers in the compute cluster.

The broker pattern does however introduce a slight overhead, which reduces the performance.\\
\textbf{Benefits \& Liabilities} ~
\begin{itemize}
\item[+] Reliability
\item[+] Modifiability
\item[+] Interoperability
\item[$-$] Performance
\end{itemize}

\subsection*{Model View Controller}
Model View Controller decouples the view, the logic, and the data modeling of the interface. Thus it is easier to modify the view without interfering the logic and data model. This results in good modifiability and better update to the usability since updating the view is relatively easy to increase the usability.

\textbf{Benefits \& Liabilities} ~
\begin{itemize} 
\item[+] Usability
\end{itemize}

\subsection*{Template view}
Template view separates each components of a web page to individual snippets. The snippets are combined together to form a web page using a HTML helper. Since the elements are separated, code reusability is increased. Updates are also easy to be implemented, since updating one snippet may results in updated elements in several pages at once. Thus, template view increases the usability of the system because of its ease of web page update.

\textbf{Benefits \& Liabilities} ~
\begin{itemize} 
\item[+] Usability
\end{itemize}

\subsection*{Unit of Work}
The Unit of Work will significantly reduce the load on the database. The increased performance has a direct impact on the usability of the system. Faster processing will make the web interface of the system more responsive to the actions of the user, which increases the usability.

While the unit of work does introduce a delay before the change in the in-memory object is present in the representation in the database, this delay is very short and does not affect the functional correctness of the system or the reliability of its results.\\
\textbf{Benefits \& Liabilities} ~
\begin{itemize}
\item[+] Performance 
\item[+] Usability
\end{itemize}


\subsection*{Shared Repository} 
%Repositories are the single point where we hand off and fetch objects. It is also the boundary where communication with the storage starts and ends.
%Mediates between the domain and the data mapping layers using a collection-like interface for accessing domain objects.
The repository gives a lot more control over how the data is handled. The benefits are:
\begin{itemize}
\item Reduces code (and code complexity)
\item Increases performance
\item Separated domain and data layers, increasing flexibility and changeability
\end{itemize}

Performing analysis on the data also consists of executing complex queries on the data source. The database that executes these queries, however, might change. Or the system might decide to use multiple databases and data sources.
Using the Repository pattern, these changes can \underline{be made fast}. The repository also allows for \underline{multiple configurations} to exist. So an extra repository could be created for testing purposes, only using an in-memory database to increase the test execution speed. \\
\textbf{Benefits \& Liabilities} ~
\begin{itemize}
\item[+] Reusability : because the data store is central
\item[+] Integrability : because the data store is central
\item[+] Maintainability : thanks to the Separation of concerns
\item[+] Modifiability : thanks to the Separation of concerns
\end{itemize}

\section{Requirements validation}
\label{sec:req-validation}
%!TEX root = ../report.tex
\subsection{Functional requirements}
\label{sec:fr-validation}
Functional requirements evaluation is depicted in Table \ref{table:eval-functional-requirements} below.

\begin{longtable}{lllL{\tw{0.2}}L{\tw{0.4}}}
    \bo{Nr.} & \bo{Priority} & \bo{Fulfilled} & \bo{Pattern(s)} & \bo{Remarks} \\ \toprule \endhead

 	% {receive-usage}{Must}
	% { The system must be able to receive electricity usage data from devices. }
    ~\ref{fr:receive-usage} 
    & Must     
    & Yes
    & ~\ref{sec:layers}
    & Receving data is handled by the layers pattern. The process starts with reception of sensor data from service layer, then the data is forwarded to domain layer to perform several failure checks. Lastly, the data is stored through data-source layer. \\ \midrule

 	% {store-usage}{Must}
	% { The system must be able to store electricity usage data. }
	~\ref{fr:store-usage} 
    & Must     
    & Yes
    & ~\ref{sec:unit-of-work}, \ref{sec:data-source-layer}, \ref{sec:repository}
    & Storing the data is mainly handled in data-source layer. This layer receives the data from the domain-layer. Unit of Work keeps track of changes to objects and coordinates storing the data to the database in one database call. The data is stored in single repository, so to say. \\ \midrule
	
	% {retrieve-usage}{Must}
	% { The system must be able to retrieve previously stored electricity usage data. }
	~\ref{fr:retrieve-usage} 
    & Must
    & Yes
    & ~\ref{sec:unit-of-work}, \ref{sec:data-source-layer}, \ref{sec:repository}
    & Previously stored data is available in the repository, which belongs to data-source layer. Unit of Work also keeps track of changes to objects. \\ \midrule
	
	% {compute-total}{Must}
	% { The system must be able to compute the total electricity consumption for a given time period for a particular house. }
	~\ref{fr:compute-total} 
    & Must     
    & Yes
    & ~\ref{sec:broker}, \ref{sec:layers}, \ref{sec:unit-of-work}
    & Computation process involves multiple pattern: domain and data-source layer. This process is also a high read-write process, thus unit of work pattern is useful for tracking the changes of objects and coordinates read/write process to the database. Broker makes sure that the parallel computation is taken care properly. \\ \midrule
	
	% {compute-timeperiod}{Must}
	% { The system must be able to compute the electricity consumption per device for a given time period. }
	~\ref{fr:compute-timeperiod} 
    & Must     
    & Yes
    & ~\ref{sec:broker}, \ref{sec:layers}, \ref{sec:unit-of-work}
    & This process is also similar with previous process (\ref{fr:compute-timeperiod}), but this is more time-range restricted. \\ \midrule
	
	% {compute-bill}{Must}
	% { The system must be able to compute an estimated electricity bill for the current month based on the electricity consumption to that point. }
	~\ref{fr:compute-bill} 
    & Must     
    & Yes
    & ~\ref{sec:broker}, \ref{sec:layers}, \ref{sec:unit-of-work}
    & Computing bill process involves multiple pattern: domain and data-source layer. This process is also a high read-write process, thus unit of work pattern is useful for tracking the changes of objects and coordinates read/write process to the database. Broker makes sure that the parallel computation is taken care properly. \\ \midrule
	
	% {interface-selectstats}{Must}
	% { A user of the system must be able to select which statistics to compute in a web interface. }
	~\ref{fr:interface-selectstats} 
    & Must     
    & Yes
    & ~\ref{sec:mvc-analysis}, \ref{sec:template-view-analysis}
    & All operations done in web interface are taken care by the template view and MVC pattern. \\ \midrule
	
	% {web-interface}{Must}
	% { The system must be able to show the computed statistics in a web interface. }
	~\ref{fr:web-interface} 
    & Must     
    & Yes
    & ~\ref{sec:mvc-analysis}, \ref{sec:template-view-analysis}
    & All operations done in web interface are taken care by the template view and MVC pattern.\\ \midrule
	
	% {interface-register}{Must}
	% { The system must allow users to register an account on the web interface. }
	~\ref{fr:interface-register} 
    & Must     
    & Yes
    & ~\ref{sec:front-page}, \ref{sec:mvc-analysis}, \ref{sec:template-view-analysis}
    & Registration, which is done through web interface, is taken care by the template view and MVC pattern as well. The front page will handle the logging, authentication and initial security of the request. \\ \midrule
	
	% {interface-login}{Must}
	% { The system must require users to be logged in, before the user can view electricity usage information about his/her house. }
	~\ref{fr:interface-login} 
    & Must     
    & Yes
    & ~\ref{sec:front-page}, \ref{sec:mvc-analysis}, \ref{sec:template-view-analysis}
    & The front page will handle the logging, authentication and initial security of the request. User interface is provided through MVC and template view pattern. \\ \midrule
	
	% {add-house}{Must}
	% { A user of the system must be able to register a new house using the web interface. }
	~\ref{fr:add-house} 
    & Must     
    & Yes
    & ~\ref{sec:front-page}, \ref{sec:mvc-analysis}, \ref{sec:template-view-analysis}
    & In order to add a house, a user must be logged in. The front page will handle the logging, authentication and initial security of the request. User interface is provided through MVC and template view pattern. \\ \midrule
	
	% {add-device}{Must}
	% { A user of the system must be able to register a new device for his house using the web interface. }
	~\ref{fr:add-device} 
    & Must
    & Yes
    & ~\ref{sec:front-page}, \ref{sec:mvc-analysis}, \ref{sec:template-view-analysis}
    & In order to add a device, a user must be logged in. The front page will handle the logging, authentication and initial security of the request. User interface is provided through MVC and template view pattern. \\ \midrule
	
	% {configure-price}{Must}
	% { A user of the system must be able to configure the price of a kWH in the web interface. }
	~\ref{fr:configure-price} 
    & Must     
    & Yes
    & ~\ref{sec:front-page}, \ref{sec:mvc-analysis}, \ref{sec:template-view-analysis}
    & In order to configure the price of the electricity, a user must be logged in. The front page will handle the logging, authentication and initial security of the request. User interface is provided through MVC and template view pattern. \\ \midrule
	
	% {feedback}{Must}
	% { The system must be able to send feedback to registered devices about the current electricity usage. }
	~\ref{fr:feedback} 
    & Must     
    & Yes
    & ~\ref{sec:layers}, \ref{sec:service-layer}
    & This is multi-tier operation, which involves all layers. The feedback is sent through the service layer. \\ \midrule
	
	% {show-accuracy}{Must}
	% { The system must be able to take the inaccuracy of the sensors into account when computing the statistics. }
	~\ref{fr:show-accuracy} 
    & Must     
    & Yes
    & ~\ref{sec:layers}, \ref{sec:domain-layer}, \ref{sec:unit-of-work}
    & The accuracy is provided by the sensor itself. The computation process is taken care in the domain layer. \\ \midrule
	
	% {show-history}{Must}
	% { The system must be able to display the history of electricity usage. }
	~\ref{fr:show-history} 
    & Must     
    & Yes
    & ~\ref{sec:mvc-analysis}, \ref{sec:template-view-analysis}
    & The history, computed in \ref{fr:retrieve-usage}, is shown to the user through web interface, which involves MVC and template view pattern. \\ \midrule
	
	% {choose-alerts}{Should}
	% { Users of the system should be able to subscribe to alerts in the web interface, alerting them about sudden energy increases or when they are using more energy than in previous months/weeks. }
	~\ref{fr:choose-alerts} 
    & Should     
    & Yes
    & ~\ref{sec:mvc-analysis}, \ref{sec:template-view-analysis}
    & The alert is shown in the web interface, which employs MVC and template view pattern. \\ \midrule
	
	% {alerts}{Should}
	% { The system should send alerts to users by mail when the user is subscribed for this alert and the condition for the alert is met. }
	~\ref{fr:alerts}
    & Should     
    & Yes
    & ~\ref{sec:layers}, \ref{sec:domain-layer}
    & This is a multi-tier operation, which includes domain and service layer. The email is sent through service layer. \\ \midrule
	
	% {interface-graph}{Should}
	% { The system should be able to show the computed statistics in a graph. }
	~\ref{fr:interface-graph} 
    & Should
    & Yes
    & ~\ref{sec:mvc-analysis}, \ref{sec:template-view-analysis}
    & Graph is shown in the web interface, which employs MVC and template view pattern. \\\bottomrule		

    \caption{Evaluation of functional-requirements}
    \label{table:eval-functional-requirements}
\end{longtable}
%!TEX root = ../report.tex
\subsection{Non-Functional requirement}
\label{sec:nfr-validation}
This subsection presents non-functional requirements evaluation in tables.

\subsubsection{Usability}

\begin{longtable}{llL{\tw{0.2}}L{\tw{0.4}}}
    \bo{Nr.} & \bo{Fulfilled} & \bo{Pattern(s)} & \bo{Remarks} \\ \toprule \endhead
	
	% \reqRow{US}{app}{An application is available for tablets and phone with those OS Windows Phone, Android, and iPhone.} \\
	~\ref{US:app}
    & Yes
    & ~\ref{sec:mvc-analysis}, \ref{sec:template-view-analysis}
    & Fulfilled with web application, which employs MVC and template view pattern. \\
	\midrule
	% \reqRow{US}{basic-understanding}{The end-user needs maximum ten minutes to get a basic understanding of system features through the UI. }\\
	~\ref{US:basic-understanding}
    & Yes
    & ~\ref{sec:mvc-analysis}, \ref{sec:template-view-analysis}
    & Fulfilled by using MVC and template view pattern. \\
	\midrule
	% \reqRow{US}{spa}{Every feature/major option of the system can be accessed through the home page (Single Page Application). }\\
	~\ref{US:spa}
    & Yes
    & ~\ref{sec:mvc-analysis}, \ref{sec:template-view-analysis}
    & Fulfilled by using MVC and template view pattern. \\
    \bottomrule		

    \caption{Evaluation of non-functional requirements: usability}
    \label{table:eval-usability}
\end{longtable}

\subsubsection{Reliability}

\begin{longtable}{llL{\tw{0.2}}L{\tw{0.4}}}
    \bo{Nr.} & \bo{Fulfilled} & \bo{Pattern(s)} & \bo{Remarks} \\ \toprule \endhead
	
	% \reqRow{RE}{error-margin}{A margin of error of $5\%$ in the energy measurements is tolerated.} \\
	~\ref{RE:error-margin}
    & No
    & -
    & Sensor accuracy is out of scope of this project as this project does not deal with sensor hardware. \\
	\midrule
	% \reqRow{RE}{availability}{The system should be available $99.9\%$. of the time which means down for 44 minutes within 6 months.}\\
	~\ref{RE:availability}
    & Yes
    & ~\ref{sec:layers}, \ref{sec:service-layer}, \ref{sec:unitofwork}, \ref{sec:broker}
    & High availability is achieved by separating the service in layers and parallelizing the processes.\\
	\midrule
	% \reqRow{RE}{downtime}{The system should be down for maximum ten minutes when the user installs a new release (version).}\\
	~\ref{RE:downtime}
    & Yes
    & ~\ref{sec:layers}, \ref{sec:service-layer}, \ref{sec:unitofwork}, \ref{sec:broker}
    & Separated layers tame care smooth new version deployment in the HEMS.\\
    \bottomrule		

    \caption{Evaluation of non-functional requirements: reliability}
    \label{table:eval-reliability}
\end{longtable}

\subsubsection{Security}

\begin{longtable}{llL{\tw{0.2}}L{\tw{0.4}}}
    \bo{Nr.} & \bo{Fulfilled} & \bo{Pattern(s)} & \bo{Remarks} \\ \toprule \endhead
	
	% \reqRow{SEC}{login}{Each user is identified and has to log in in order to access his "Home Energy Monitor" account.} \\
	~\ref{SEC:login}
    & Yes
    & ~\ref{sec:front-page}, \ref{sec:mvc-analysis}, \ref{sec:template-view-analysis}
    & Login page is handled by MVC and template view pattern. Front page pattern also helps to serve the login request. \\
	\midrule
	% \reqRow{SEC}{encryption}{The data stored in the database is encrypted.}\\
	~\ref{SEC:encryption}
    & Yes
    & ~\ref{sec:repository}
    & Repository pattern is used to encrypt the data in the database.\\
	\midrule
	% \reqRow{SEC}{https}{The connection from/to the system is encrypted using secure connection.}\\
	~\ref{SEC:https}
    & Yes
    & ~\ref{sec:layers}, \ref{sec:service-layer}
    & Layers and service layer pattern make sure the connections are secured using HTTPS channel. \\
    \bottomrule		

    \caption{Evaluation of non-functional requirements: security}
    \label{table:eval-security}
\end{longtable}

\subsubsection{Interoperability}

\begin{longtable}{llL{\tw{0.2}}L{\tw{0.4}}}
    \bo{Nr.} & \bo{Fulfilled} & \bo{Pattern(s)} & \bo{Remarks} \\ \toprule \endhead
	
	% \reqRow{INT}{interface}{The web interface of the system is available and functioning for 95\% of the browser market share.} \\
	~\ref{INT:interface}
    & Yes
    & ~\ref{sec:mvc-analysis}, \ref{sec:template-view-analysis}
    & The web interface are mainly handled by MVC and template view pattern.\\
	\midrule
	% \reqRow{INT}{rest}{The system exposes a REST interface that allows different electricity usage sensors to submit the electricity usage data.}\\
	~\ref{INT:rest}
    & Yes
    & ~\ref{sec:layers}, \ref{sec:service-layer}
    & REST interface are exposed in service layer. \\
    \bottomrule		

    \caption{Evaluation of non-functional requirements: interoperability}
    \label{table:eval-interoperability}
\end{longtable}

\subsubsection{Scalability}

\begin{longtable}{llL{\tw{0.2}}L{\tw{0.4}}}
    \bo{Nr.} & \bo{Fulfilled} & \bo{Pattern(s)} & \bo{Remarks} \\ \toprule \endhead
	
	% \reqRow{SCA}{user}{The system should still perform as efficiently as it is suppose to be when the number of users doubled.} \\
	~\ref{SCA:user}
    & Yes
    & ~\ref{sec:layers}, \ref{sec:broker}, \ref{sec:unitofwork}, \ref{sec:datamapper}
    & The system uses layers which are possible to be replicated.\\
	\midrule
	% \reqRow{SCA}{device}{The system should still perform as efficiently as it is suppose to be when the number of devices doubled.}\\
	~\ref{SCA:device}
    & Yes
    & ~\ref{sec:layers}, \ref{sec:broker}, \ref{sec:unitofwork}, \ref{sec:datamapper}
    & The system uses layers which are possible to be replicated.\\
    \bottomrule		

    \caption{Evaluation of non-functional requirements: scalability}
    \label{table:eval-scalability}
\end{longtable}

